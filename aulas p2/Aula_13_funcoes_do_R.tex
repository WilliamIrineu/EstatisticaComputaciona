% Options for packages loaded elsewhere
\PassOptionsToPackage{unicode}{hyperref}
\PassOptionsToPackage{hyphens}{url}
%
\documentclass[
]{article}
\usepackage{amsmath,amssymb}
\usepackage{iftex}
\ifPDFTeX
  \usepackage[T1]{fontenc}
  \usepackage[utf8]{inputenc}
  \usepackage{textcomp} % provide euro and other symbols
\else % if luatex or xetex
  \usepackage{unicode-math} % this also loads fontspec
  \defaultfontfeatures{Scale=MatchLowercase}
  \defaultfontfeatures[\rmfamily]{Ligatures=TeX,Scale=1}
\fi
\usepackage{lmodern}
\ifPDFTeX\else
  % xetex/luatex font selection
\fi
% Use upquote if available, for straight quotes in verbatim environments
\IfFileExists{upquote.sty}{\usepackage{upquote}}{}
\IfFileExists{microtype.sty}{% use microtype if available
  \usepackage[]{microtype}
  \UseMicrotypeSet[protrusion]{basicmath} % disable protrusion for tt fonts
}{}
\makeatletter
\@ifundefined{KOMAClassName}{% if non-KOMA class
  \IfFileExists{parskip.sty}{%
    \usepackage{parskip}
  }{% else
    \setlength{\parindent}{0pt}
    \setlength{\parskip}{6pt plus 2pt minus 1pt}}
}{% if KOMA class
  \KOMAoptions{parskip=half}}
\makeatother
\usepackage{xcolor}
\usepackage[margin=1in]{geometry}
\usepackage{color}
\usepackage{fancyvrb}
\newcommand{\VerbBar}{|}
\newcommand{\VERB}{\Verb[commandchars=\\\{\}]}
\DefineVerbatimEnvironment{Highlighting}{Verbatim}{commandchars=\\\{\}}
% Add ',fontsize=\small' for more characters per line
\usepackage{framed}
\definecolor{shadecolor}{RGB}{248,248,248}
\newenvironment{Shaded}{\begin{snugshade}}{\end{snugshade}}
\newcommand{\AlertTok}[1]{\textcolor[rgb]{0.94,0.16,0.16}{#1}}
\newcommand{\AnnotationTok}[1]{\textcolor[rgb]{0.56,0.35,0.01}{\textbf{\textit{#1}}}}
\newcommand{\AttributeTok}[1]{\textcolor[rgb]{0.13,0.29,0.53}{#1}}
\newcommand{\BaseNTok}[1]{\textcolor[rgb]{0.00,0.00,0.81}{#1}}
\newcommand{\BuiltInTok}[1]{#1}
\newcommand{\CharTok}[1]{\textcolor[rgb]{0.31,0.60,0.02}{#1}}
\newcommand{\CommentTok}[1]{\textcolor[rgb]{0.56,0.35,0.01}{\textit{#1}}}
\newcommand{\CommentVarTok}[1]{\textcolor[rgb]{0.56,0.35,0.01}{\textbf{\textit{#1}}}}
\newcommand{\ConstantTok}[1]{\textcolor[rgb]{0.56,0.35,0.01}{#1}}
\newcommand{\ControlFlowTok}[1]{\textcolor[rgb]{0.13,0.29,0.53}{\textbf{#1}}}
\newcommand{\DataTypeTok}[1]{\textcolor[rgb]{0.13,0.29,0.53}{#1}}
\newcommand{\DecValTok}[1]{\textcolor[rgb]{0.00,0.00,0.81}{#1}}
\newcommand{\DocumentationTok}[1]{\textcolor[rgb]{0.56,0.35,0.01}{\textbf{\textit{#1}}}}
\newcommand{\ErrorTok}[1]{\textcolor[rgb]{0.64,0.00,0.00}{\textbf{#1}}}
\newcommand{\ExtensionTok}[1]{#1}
\newcommand{\FloatTok}[1]{\textcolor[rgb]{0.00,0.00,0.81}{#1}}
\newcommand{\FunctionTok}[1]{\textcolor[rgb]{0.13,0.29,0.53}{\textbf{#1}}}
\newcommand{\ImportTok}[1]{#1}
\newcommand{\InformationTok}[1]{\textcolor[rgb]{0.56,0.35,0.01}{\textbf{\textit{#1}}}}
\newcommand{\KeywordTok}[1]{\textcolor[rgb]{0.13,0.29,0.53}{\textbf{#1}}}
\newcommand{\NormalTok}[1]{#1}
\newcommand{\OperatorTok}[1]{\textcolor[rgb]{0.81,0.36,0.00}{\textbf{#1}}}
\newcommand{\OtherTok}[1]{\textcolor[rgb]{0.56,0.35,0.01}{#1}}
\newcommand{\PreprocessorTok}[1]{\textcolor[rgb]{0.56,0.35,0.01}{\textit{#1}}}
\newcommand{\RegionMarkerTok}[1]{#1}
\newcommand{\SpecialCharTok}[1]{\textcolor[rgb]{0.81,0.36,0.00}{\textbf{#1}}}
\newcommand{\SpecialStringTok}[1]{\textcolor[rgb]{0.31,0.60,0.02}{#1}}
\newcommand{\StringTok}[1]{\textcolor[rgb]{0.31,0.60,0.02}{#1}}
\newcommand{\VariableTok}[1]{\textcolor[rgb]{0.00,0.00,0.00}{#1}}
\newcommand{\VerbatimStringTok}[1]{\textcolor[rgb]{0.31,0.60,0.02}{#1}}
\newcommand{\WarningTok}[1]{\textcolor[rgb]{0.56,0.35,0.01}{\textbf{\textit{#1}}}}
\usepackage{graphicx}
\makeatletter
\def\maxwidth{\ifdim\Gin@nat@width>\linewidth\linewidth\else\Gin@nat@width\fi}
\def\maxheight{\ifdim\Gin@nat@height>\textheight\textheight\else\Gin@nat@height\fi}
\makeatother
% Scale images if necessary, so that they will not overflow the page
% margins by default, and it is still possible to overwrite the defaults
% using explicit options in \includegraphics[width, height, ...]{}
\setkeys{Gin}{width=\maxwidth,height=\maxheight,keepaspectratio}
% Set default figure placement to htbp
\makeatletter
\def\fps@figure{htbp}
\makeatother
\setlength{\emergencystretch}{3em} % prevent overfull lines
\providecommand{\tightlist}{%
  \setlength{\itemsep}{0pt}\setlength{\parskip}{0pt}}
\setcounter{secnumdepth}{-\maxdimen} % remove section numbering
\ifLuaTeX
  \usepackage{selnolig}  % disable illegal ligatures
\fi
\IfFileExists{bookmark.sty}{\usepackage{bookmark}}{\usepackage{hyperref}}
\IfFileExists{xurl.sty}{\usepackage{xurl}}{} % add URL line breaks if available
\urlstyle{same}
\hypersetup{
  pdftitle={Aula 13: Otimização numérica - Aspectos Computacionais},
  pdfauthor={Prof.~Dr.~Eder Angelo Milani},
  hidelinks,
  pdfcreator={LaTeX via pandoc}}

\title{Aula 13: Otimização numérica - Aspectos Computacionais}
\author{Prof.~Dr.~Eder Angelo Milani}
\date{19/06/2023}

\begin{document}
\maketitle

A seguir é apresentado algumas funções interessantes no R que podem
ajudar na difícil tarefa de otimização de funções.

\hypertarget{funuxe7uxf5es-hessian-e-grad-no-r}{%
\section{Funções hessian e grad no
R}\label{funuxe7uxf5es-hessian-e-grad-no-r}}

O método de Newton-Raphson, que além do gradiente, usa o hessiano e é
apropriado para calcular estimadores de máxima verossimilhança por meio
de uma aproximação quadrática da verossimilhança ao redor de valores
iniciais dos parâmetros, apresenta a dificuldade dos cálculos teóricos.

Uma alternativa é a obtenção dos valores necessários por meio de
aproximações numéricas. Veja o exemplo a seguir.

\hypertarget{exemplo-1}{%
\subsection{Exemplo 1}\label{exemplo-1}}

Voltando ao Exemplo 2 da Aula 10, foi gerado uma amostra da distribuição
Weibull(\(\alpha, \gamma\)), na sequência foi obtido as funções de
verossimilhança, log-verossimilhança, vetor escore e matriz hessiana.

Podemos utilizar funções no R que calculam, de forma aproximada, o vetor
escore e a matriz hessiana. Portanto, facilitando o processo como um
todo.

\begin{Shaded}
\begin{Highlighting}[]
\FunctionTok{set.seed}\NormalTok{(}\DecValTok{2023}\NormalTok{)}
\NormalTok{n }\OtherTok{\textless{}{-}} \DecValTok{1000}
\NormalTok{alpha }\OtherTok{\textless{}{-}} \DecValTok{2}
\NormalTok{gama }\OtherTok{\textless{}{-}} \FloatTok{0.5}

\NormalTok{x}\OtherTok{\textless{}{-}} \FunctionTok{rweibull}\NormalTok{(n, }\AttributeTok{shape =}\NormalTok{gama , }\AttributeTok{scale =}\NormalTok{ alpha)}

\FunctionTok{summary}\NormalTok{(x)}
\end{Highlighting}
\end{Shaded}

\begin{verbatim}
##     Min.  1st Qu.   Median     Mean  3rd Qu.     Max. 
##   0.0000   0.1802   1.1010   3.9462   4.1123 125.5789
\end{verbatim}

\begin{Shaded}
\begin{Highlighting}[]
\CommentTok{\# o valor esperado e}
\NormalTok{alpha}\SpecialCharTok{*}\FunctionTok{gamma}\NormalTok{(}\DecValTok{1}\SpecialCharTok{+}\DecValTok{1}\SpecialCharTok{/}\NormalTok{gama)}
\end{Highlighting}
\end{Shaded}

\begin{verbatim}
## [1] 4
\end{verbatim}

\begin{Shaded}
\begin{Highlighting}[]
\CommentTok{\# construir a funcao objeto }

\NormalTok{log\_vero }\OtherTok{\textless{}{-}} \ControlFlowTok{function}\NormalTok{(p0)\{}
\NormalTok{  alpha }\OtherTok{\textless{}{-}}\NormalTok{ p0[}\DecValTok{1}\NormalTok{]}
\NormalTok{  gama }\OtherTok{\textless{}{-}}\NormalTok{ p0[}\DecValTok{2}\NormalTok{]}
\NormalTok{  n }\OtherTok{\textless{}{-}} \FunctionTok{length}\NormalTok{(x)}
\NormalTok{  aux }\OtherTok{\textless{}{-}}\NormalTok{ n}\SpecialCharTok{*}\FunctionTok{log}\NormalTok{(gama) }\SpecialCharTok{{-}}\NormalTok{ n}\SpecialCharTok{*}\NormalTok{gama}\SpecialCharTok{*}\FunctionTok{log}\NormalTok{(alpha) }\SpecialCharTok{+}\NormalTok{ (gama}\DecValTok{{-}1}\NormalTok{)}\SpecialCharTok{*}\FunctionTok{sum}\NormalTok{(}\FunctionTok{log}\NormalTok{(x)) }\SpecialCharTok{{-}} \FunctionTok{sum}\NormalTok{((x}\SpecialCharTok{/}\NormalTok{alpha)}\SpecialCharTok{\^{}}\NormalTok{gama)}
  \FunctionTok{return}\NormalTok{(aux)}
\NormalTok{\}}

\CommentTok{\# teste da funcao }
\FunctionTok{log\_vero}\NormalTok{(}\AttributeTok{p0=}\FunctionTok{c}\NormalTok{(}\DecValTok{1}\NormalTok{,}\FloatTok{0.5}\NormalTok{))}
\end{Highlighting}
\end{Shaded}

\begin{verbatim}
## [1] -1938.947
\end{verbatim}

\begin{Shaded}
\begin{Highlighting}[]
\CommentTok{\# pacote a ser carregado}
\FunctionTok{library}\NormalTok{(numDeriv)}

\CommentTok{\# o vetor escore }
\FunctionTok{grad}\NormalTok{(log\_vero, }\AttributeTok{x=}\FunctionTok{c}\NormalTok{(}\DecValTok{1}\NormalTok{,}\FloatTok{0.5}\NormalTok{))}
\end{Highlighting}
\end{Shaded}

\begin{verbatim}
## [1]  217.3866 -545.4308
\end{verbatim}

\begin{Shaded}
\begin{Highlighting}[]
\CommentTok{\# a matriz hessian}
\FunctionTok{hessian}\NormalTok{(log\_vero, }\AttributeTok{x=}\FunctionTok{c}\NormalTok{(}\DecValTok{1}\NormalTok{,}\FloatTok{0.5}\NormalTok{))}
\end{Highlighting}
\end{Shaded}

\begin{verbatim}
##          [,1]       [,2]
## [1,] -576.080   1518.516
## [2,] 1518.516 -10737.908
\end{verbatim}

\begin{Shaded}
\begin{Highlighting}[]
\CommentTok{\# metodo de newton{-}raphson}

\NormalTok{n.max }\OtherTok{\textless{}{-}} \DecValTok{100}
\NormalTok{theta }\OtherTok{\textless{}{-}} \FunctionTok{matrix}\NormalTok{(}\ConstantTok{NA}\NormalTok{, }\AttributeTok{nrow=}\DecValTok{2}\NormalTok{, }\AttributeTok{ncol=}\NormalTok{n.max)}
\NormalTok{dif\_ }\OtherTok{\textless{}{-}} \DecValTok{1}
\NormalTok{epsilon }\OtherTok{\textless{}{-}} \FloatTok{0.001}
\NormalTok{cont }\OtherTok{\textless{}{-}} \DecValTok{1}
\NormalTok{theta[, }\DecValTok{1}\NormalTok{] }\OtherTok{\textless{}{-}} \FunctionTok{c}\NormalTok{(}\DecValTok{1}\NormalTok{, }\DecValTok{2}\NormalTok{)}

\ControlFlowTok{while}\NormalTok{(}\FunctionTok{abs}\NormalTok{(dif\_)}\SpecialCharTok{\textgreater{}=}\NormalTok{ epsilon }\SpecialCharTok{\&}\NormalTok{ cont}\SpecialCharTok{\textless{}=}\NormalTok{n.max)\{}
  \FunctionTok{cat}\NormalTok{(}\StringTok{"cont+1="}\NormalTok{, cont}\SpecialCharTok{+}\DecValTok{1}\NormalTok{, }\StringTok{"}\SpecialCharTok{\textbackslash{}n}\StringTok{"}\NormalTok{)}
\NormalTok{  theta[ ,cont}\SpecialCharTok{+}\DecValTok{1}\NormalTok{] }\OtherTok{\textless{}{-}}\NormalTok{ theta[ ,cont] }\SpecialCharTok{{-}} \FunctionTok{solve}\NormalTok{(}\FunctionTok{hessian}\NormalTok{(log\_vero, theta[ ,cont]))}\SpecialCharTok{\%*\%}\FunctionTok{grad}\NormalTok{(log\_vero, theta[ ,cont])}
  \FunctionTok{cat}\NormalTok{(}\StringTok{"theta["}\NormalTok{, cont}\SpecialCharTok{+}\DecValTok{1}\NormalTok{, }\StringTok{"]="}\NormalTok{, theta[ ,cont}\SpecialCharTok{+}\DecValTok{1}\NormalTok{], }\StringTok{"}\SpecialCharTok{\textbackslash{}n}\StringTok{"}\NormalTok{)}
\NormalTok{  dif\_ }\OtherTok{\textless{}{-}} \FunctionTok{sqrt}\NormalTok{(}\FunctionTok{sum}\NormalTok{((theta[ ,cont}\SpecialCharTok{+}\DecValTok{1}\NormalTok{]}\SpecialCharTok{{-}}\NormalTok{theta[ ,cont])}\SpecialCharTok{\^{}}\DecValTok{2}\NormalTok{))}
  \FunctionTok{cat}\NormalTok{(}\StringTok{"dif="}\NormalTok{, dif\_, }\StringTok{"}\SpecialCharTok{\textbackslash{}n}\StringTok{"}\NormalTok{)}
\NormalTok{  cont }\OtherTok{\textless{}{-}}\NormalTok{ cont}\SpecialCharTok{+}\DecValTok{1}
\NormalTok{\}}
\end{Highlighting}
\end{Shaded}

\begin{verbatim}
## cont+1= 2 
## theta[ 2 ]= 0.8781681 1.677699 
## dif= 0.3445592 
## cont+1= 3 
## theta[ 3 ]= 0.7362294 1.334386 
## dif= 0.3714973 
## cont+1= 4 
## theta[ 4 ]= 0.5547017 0.9573204 
## dif= 0.4184863 
## cont+1= 5 
## theta[ 5 ]= 0.3563464 0.5738512 
## dif= 0.4317331 
## cont+1= 6 
## theta[ 6 ]= 0.3312675 0.381048 
## dif= 0.1944274 
## cont+1= 7 
## theta[ 7 ]= 0.5641077 0.4181372 
## dif= 0.2357757 
## cont+1= 8 
## theta[ 8 ]= 0.9305036 0.4671185 
## dif= 0.3696554 
## cont+1= 9 
## theta[ 9 ]= 1.378278 0.501388 
## dif= 0.4490838 
## cont+1= 10 
## theta[ 10 ]= 1.790903 0.5153879 
## dif= 0.412862 
## cont+1= 11 
## theta[ 11 ]= 2.041855 0.5178614 
## dif= 0.2509642 
## cont+1= 12 
## theta[ 12 ]= 2.112259 0.5178338 
## dif= 0.07040396 
## cont+1= 13 
## theta[ 13 ]= 2.116738 0.5178167 
## dif= 0.004479602 
## cont+1= 14 
## theta[ 14 ]= 2.116755 0.5178166 
## dif= 1.699994e-05
\end{verbatim}

\begin{Shaded}
\begin{Highlighting}[]
\NormalTok{cont}
\end{Highlighting}
\end{Shaded}

\begin{verbatim}
## [1] 14
\end{verbatim}

\begin{Shaded}
\begin{Highlighting}[]
\NormalTok{theta[, }\DecValTok{1}\SpecialCharTok{:}\NormalTok{cont]}
\end{Highlighting}
\end{Shaded}

\begin{verbatim}
##      [,1]      [,2]      [,3]      [,4]      [,5]      [,6]      [,7]      [,8]
## [1,]    1 0.8781681 0.7362294 0.5547017 0.3563464 0.3312675 0.5641077 0.9305036
## [2,]    2 1.6776988 1.3343860 0.9573204 0.5738512 0.3810480 0.4181372 0.4671185
##          [,9]     [,10]     [,11]     [,12]     [,13]     [,14]
## [1,] 1.378278 1.7909026 2.0418546 2.1122585 2.1167381 2.1167551
## [2,] 0.501388 0.5153879 0.5178614 0.5178338 0.5178167 0.5178166
\end{verbatim}

Compare os resultados obtidos aqui com os resultados da Aula 10.

\hypertarget{funuxe7uxf5es-optim-e-maxlik}{%
\section{Funções optim e MaxLik}\label{funuxe7uxf5es-optim-e-maxlik}}

Para obter máximos ou mínimos de funções pode-se usar a função
\texttt{optim()} do pacote \texttt{stats}, que inclui vários métodos de
otimização, dentre os quais destacamos:

\begin{enumerate}
\def\labelenumi{\roman{enumi})}
\item
  o método default que é o Método Simplex de Nelder e Mead (1965), e
  corresponde a um método de busca não gradiente,
\item
  o método BFGS,
\item
  o método do gradiente conjugado (conjugate gradient - CG),
\item
  o método L-BFGS-B, que permite restrições limitadas,
\item
  o método SANN, que é método não gradiente e pode ser encarado como uma
  variante do método da têmpera simulada (simulated annealing).
\end{enumerate}

Outra opção é o pacote \texttt{optimization}, que também inclui vários
métodos, como o método Nelder-Mead e da têmpera simulada. O pacote
\texttt{maxLik} também é uma opção, contendo quase todas as funções do
\texttt{stats}, além do algoritmo de \texttt{Newton–Rapson} (função
\texttt{maxNR()}). Esse pacote é apropriado para a maximização de
verossimilhanças, daí o rótulo maxLik.

\hypertarget{exemplo-2}{%
\subsection{Exemplo 2}\label{exemplo-2}}

Considere a função \(f(x,y)=\exp(-(x+ y))\), que tem um máximo no ponto
(0, 0) e valor máximo igual a 1. A seguir é apresentado a função
\texttt{maxNR()} para obter o ponto de máximo.

\begin{Shaded}
\begin{Highlighting}[]
\CommentTok{\#install.packages("maxLik")}
\FunctionTok{library}\NormalTok{(}\StringTok{"maxLik"}\NormalTok{)}
\end{Highlighting}
\end{Shaded}

\begin{verbatim}
## Warning: package 'maxLik' was built under R version 4.3.1
\end{verbatim}

\begin{verbatim}
## Carregando pacotes exigidos: miscTools
\end{verbatim}

\begin{verbatim}
## Warning: package 'miscTools' was built under R version 4.3.1
\end{verbatim}

\begin{verbatim}
## 
## Please cite the 'maxLik' package as:
## Henningsen, Arne and Toomet, Ott (2011). maxLik: A package for maximum likelihood estimation in R. Computational Statistics 26(3), 443-458. DOI 10.1007/s00180-010-0217-1.
## 
## If you have questions, suggestions, or comments regarding the 'maxLik' package, please use a forum or 'tracker' at maxLik's R-Forge site:
## https://r-forge.r-project.org/projects/maxlik/
\end{verbatim}

\begin{verbatim}
## 
## Attaching package: 'maxLik'
\end{verbatim}

\begin{verbatim}
## The following object is masked from 'package:numDeriv':
## 
##     hessian
\end{verbatim}

\begin{Shaded}
\begin{Highlighting}[]
\NormalTok{f1 }\OtherTok{\textless{}{-}} \ControlFlowTok{function}\NormalTok{(theta)\{ }
\NormalTok{x }\OtherTok{\textless{}{-}}\NormalTok{ theta[}\DecValTok{1}\NormalTok{]}
\NormalTok{y }\OtherTok{\textless{}{-}}\NormalTok{ theta[}\DecValTok{2}\NormalTok{]}
\NormalTok{z }\OtherTok{\textless{}{-}} \FunctionTok{exp}\NormalTok{(}\SpecialCharTok{{-}}\NormalTok{(x}\SpecialCharTok{\^{}}\DecValTok{2}\SpecialCharTok{+}\NormalTok{y}\SpecialCharTok{\^{}}\DecValTok{2}\NormalTok{))}
\FunctionTok{return}\NormalTok{(z)}
\NormalTok{\}}

\NormalTok{result }\OtherTok{\textless{}{-}} \FunctionTok{maxNR}\NormalTok{(f1, }\AttributeTok{start=}\FunctionTok{c}\NormalTok{(}\DecValTok{1}\NormalTok{,}\DecValTok{1}\NormalTok{))}

\FunctionTok{print}\NormalTok{(}\FunctionTok{summary}\NormalTok{(result))}
\end{Highlighting}
\end{Shaded}

\begin{verbatim}
## --------------------------------------------
## Newton-Raphson maximisation 
## Number of iterations: 3 
## Return code: 1 
## gradient close to zero (gradtol) 
## Function value: 1 
## Estimates:
##          estimate gradient
## [1,] 1.044902e-11        0
## [2,] 1.044902e-11        0
## --------------------------------------------
\end{verbatim}

\hypertarget{exemplo-3}{%
\subsection{Exemplo 3}\label{exemplo-3}}

Uma função comumente usada como teste em otimização é a função de
Rosenbrock, ou função banana

\[f(x,y)=(1-x)^2+100(y-x^2)^2,\]

Essa função tem um mínimo global em (1, 1). Os resultados da aplicação
da função \texttt{optim()} para determinar o mínimo dessa função, usando
diferentes métodos são apresentados a seguir.

\begin{Shaded}
\begin{Highlighting}[]
\NormalTok{f1 }\OtherTok{\textless{}{-}} \ControlFlowTok{function}\NormalTok{(theta)\{ }
\NormalTok{x }\OtherTok{\textless{}{-}}\NormalTok{ theta[}\DecValTok{1}\NormalTok{]}
\NormalTok{y }\OtherTok{\textless{}{-}}\NormalTok{ theta[}\DecValTok{2}\NormalTok{]}
\CommentTok{\#cat("x=", x, "y=", y, "\textbackslash{}n")}
\NormalTok{z }\OtherTok{\textless{}{-}}\NormalTok{ (}\DecValTok{1}\SpecialCharTok{{-}}\NormalTok{x)}\SpecialCharTok{\^{}}\DecValTok{2} \SpecialCharTok{+} \DecValTok{100}\SpecialCharTok{*}\NormalTok{(y}\SpecialCharTok{{-}}\NormalTok{x}\SpecialCharTok{\^{}}\DecValTok{2}\NormalTok{)}\SpecialCharTok{\^{}}\DecValTok{2}
\FunctionTok{return}\NormalTok{(z)}
\NormalTok{\}}

\NormalTok{chute }\OtherTok{\textless{}{-}} \FunctionTok{c}\NormalTok{(}\SpecialCharTok{{-}}\FloatTok{1.2}\NormalTok{, }\DecValTok{1}\NormalTok{)}

\CommentTok{\# BFGS}

\FunctionTok{optim}\NormalTok{(chute, f1, }\AttributeTok{method=}\StringTok{"BFGS"}\NormalTok{)}
\end{Highlighting}
\end{Shaded}

\begin{verbatim}
## $par
## [1] 0.9998044 0.9996084
## 
## $value
## [1] 3.827383e-08
## 
## $counts
## function gradient 
##      118       38 
## 
## $convergence
## [1] 0
## 
## $message
## NULL
\end{verbatim}

\begin{Shaded}
\begin{Highlighting}[]
\CommentTok{\# L{-}BFGS{-}B}

\FunctionTok{optim}\NormalTok{(chute, f1, }\AttributeTok{method=}\StringTok{"L{-}BFGS{-}B"}\NormalTok{)}
\end{Highlighting}
\end{Shaded}

\begin{verbatim}
## $par
## [1] 0.9998000 0.9996001
## 
## $value
## [1] 3.998487e-08
## 
## $counts
## function gradient 
##       49       49 
## 
## $convergence
## [1] 0
## 
## $message
## [1] "CONVERGENCE: REL_REDUCTION_OF_F <= FACTR*EPSMCH"
\end{verbatim}

\begin{Shaded}
\begin{Highlighting}[]
\CommentTok{\# CG}

\FunctionTok{optim}\NormalTok{(chute, f1, }\AttributeTok{method=}\StringTok{"CG"}\NormalTok{)}
\end{Highlighting}
\end{Shaded}

\begin{verbatim}
## $par
## [1] -0.7648079  0.5927148
## 
## $value
## [1] 3.106475
## 
## $counts
## function gradient 
##      402      101 
## 
## $convergence
## [1] 1
## 
## $message
## NULL
\end{verbatim}

\begin{Shaded}
\begin{Highlighting}[]
\CommentTok{\# Nelder{-}Mead}

\FunctionTok{optim}\NormalTok{(chute, f1, }\AttributeTok{method=}\StringTok{"Nelder{-}Mead"}\NormalTok{)}
\end{Highlighting}
\end{Shaded}

\begin{verbatim}
## $par
## [1] 1.000260 1.000506
## 
## $value
## [1] 8.825241e-08
## 
## $counts
## function gradient 
##      195       NA 
## 
## $convergence
## [1] 0
## 
## $message
## NULL
\end{verbatim}

Exercícios:

\textbf{(Entregar na sala de aula)} Gerar 100 valores da distribuição
normal(\(\mu=3, \sigma^2=4\)). Utilizando o método de Newton-Rapson,
como feito anteriormente, obtenha as estimativas de máxima
verossimilhança dos parâmetros \(\mu\) e \(\sigma^2\). Obtenha também a
estimativa adotando a função \emph{optim} e compare os resultados.

\begin{enumerate}
\def\labelenumi{(\arabic{enumi})}
\item
  Gerar 100 valores da distribuição Poisson(\(\lambda=3\)). Utilizando o
  método de Newton-Rapson, como feito anteriormente, obtenha a
  estimativa de máxima verossimilhança do parâmetro \(\lambda\). Obtenha
  também a estimativa adotando a função \emph{optim} e compare os
  resultados.
\item
  Gerar 300 valores da distribuição Bernoulli(\(p=0,3\)). Utilizando o
  método de Newton-Rapson, como feito anteriormente, obtenha a
  estimativa de máxima verossimilhança do parâmetro \(p\). Obtenha
  também a estimativa adotando a função \emph{optim} e compare os
  resultados.
\item
  Gerar 100 valores da distribuição Exp(\(\lambda=0.3\)). Utilizando o
  método de Newton-Rapson, como feito anteriormente, obtenha a
  estimativa de máxima verossimilhança do parâmetro \(\lambda\). Obtenha
  também a estimativa adotando a função \emph{optim} e compare os
  resultados.
\item
  Gerar 100 valores da distribuição gama(\(\alpha=10, \beta=10\)).
  Utilizando o método de Newton-Rapson, como feito anteriormente,
  obtenha as estimativas de máxima verossimilhança dos parâmetros
  \(\alpha\) e \(\beta\). Obtenha também a estimativa adotando a função
  \emph{optim} e compare os resultados.
\item
  Gerar 100 valores da distribuição beta(\(\alpha=10, \beta=10\)).
  Utilizando o método de Newton-Rapson, como feito anteriormente,
  obtenha as estimativas de máxima verossimilhança dos parâmetros
  \(\alpha\) e \(\beta\). Obtenha também a estimativa adotando a função
  \emph{optim} e compare os resultados.
\end{enumerate}

\end{document}
