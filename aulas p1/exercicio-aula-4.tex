% Options for packages loaded elsewhere
\PassOptionsToPackage{unicode}{hyperref}
\PassOptionsToPackage{hyphens}{url}
%
\documentclass[
]{article}
\usepackage{amsmath,amssymb}
\usepackage{lmodern}
\usepackage{iftex}
\ifPDFTeX
  \usepackage[T1]{fontenc}
  \usepackage[utf8]{inputenc}
  \usepackage{textcomp} % provide euro and other symbols
\else % if luatex or xetex
  \usepackage{unicode-math}
  \defaultfontfeatures{Scale=MatchLowercase}
  \defaultfontfeatures[\rmfamily]{Ligatures=TeX,Scale=1}
\fi
% Use upquote if available, for straight quotes in verbatim environments
\IfFileExists{upquote.sty}{\usepackage{upquote}}{}
\IfFileExists{microtype.sty}{% use microtype if available
  \usepackage[]{microtype}
  \UseMicrotypeSet[protrusion]{basicmath} % disable protrusion for tt fonts
}{}
\makeatletter
\@ifundefined{KOMAClassName}{% if non-KOMA class
  \IfFileExists{parskip.sty}{%
    \usepackage{parskip}
  }{% else
    \setlength{\parindent}{0pt}
    \setlength{\parskip}{6pt plus 2pt minus 1pt}}
}{% if KOMA class
  \KOMAoptions{parskip=half}}
\makeatother
\usepackage{xcolor}
\usepackage[margin=1in]{geometry}
\usepackage{color}
\usepackage{fancyvrb}
\newcommand{\VerbBar}{|}
\newcommand{\VERB}{\Verb[commandchars=\\\{\}]}
\DefineVerbatimEnvironment{Highlighting}{Verbatim}{commandchars=\\\{\}}
% Add ',fontsize=\small' for more characters per line
\usepackage{framed}
\definecolor{shadecolor}{RGB}{248,248,248}
\newenvironment{Shaded}{\begin{snugshade}}{\end{snugshade}}
\newcommand{\AlertTok}[1]{\textcolor[rgb]{0.94,0.16,0.16}{#1}}
\newcommand{\AnnotationTok}[1]{\textcolor[rgb]{0.56,0.35,0.01}{\textbf{\textit{#1}}}}
\newcommand{\AttributeTok}[1]{\textcolor[rgb]{0.77,0.63,0.00}{#1}}
\newcommand{\BaseNTok}[1]{\textcolor[rgb]{0.00,0.00,0.81}{#1}}
\newcommand{\BuiltInTok}[1]{#1}
\newcommand{\CharTok}[1]{\textcolor[rgb]{0.31,0.60,0.02}{#1}}
\newcommand{\CommentTok}[1]{\textcolor[rgb]{0.56,0.35,0.01}{\textit{#1}}}
\newcommand{\CommentVarTok}[1]{\textcolor[rgb]{0.56,0.35,0.01}{\textbf{\textit{#1}}}}
\newcommand{\ConstantTok}[1]{\textcolor[rgb]{0.00,0.00,0.00}{#1}}
\newcommand{\ControlFlowTok}[1]{\textcolor[rgb]{0.13,0.29,0.53}{\textbf{#1}}}
\newcommand{\DataTypeTok}[1]{\textcolor[rgb]{0.13,0.29,0.53}{#1}}
\newcommand{\DecValTok}[1]{\textcolor[rgb]{0.00,0.00,0.81}{#1}}
\newcommand{\DocumentationTok}[1]{\textcolor[rgb]{0.56,0.35,0.01}{\textbf{\textit{#1}}}}
\newcommand{\ErrorTok}[1]{\textcolor[rgb]{0.64,0.00,0.00}{\textbf{#1}}}
\newcommand{\ExtensionTok}[1]{#1}
\newcommand{\FloatTok}[1]{\textcolor[rgb]{0.00,0.00,0.81}{#1}}
\newcommand{\FunctionTok}[1]{\textcolor[rgb]{0.00,0.00,0.00}{#1}}
\newcommand{\ImportTok}[1]{#1}
\newcommand{\InformationTok}[1]{\textcolor[rgb]{0.56,0.35,0.01}{\textbf{\textit{#1}}}}
\newcommand{\KeywordTok}[1]{\textcolor[rgb]{0.13,0.29,0.53}{\textbf{#1}}}
\newcommand{\NormalTok}[1]{#1}
\newcommand{\OperatorTok}[1]{\textcolor[rgb]{0.81,0.36,0.00}{\textbf{#1}}}
\newcommand{\OtherTok}[1]{\textcolor[rgb]{0.56,0.35,0.01}{#1}}
\newcommand{\PreprocessorTok}[1]{\textcolor[rgb]{0.56,0.35,0.01}{\textit{#1}}}
\newcommand{\RegionMarkerTok}[1]{#1}
\newcommand{\SpecialCharTok}[1]{\textcolor[rgb]{0.00,0.00,0.00}{#1}}
\newcommand{\SpecialStringTok}[1]{\textcolor[rgb]{0.31,0.60,0.02}{#1}}
\newcommand{\StringTok}[1]{\textcolor[rgb]{0.31,0.60,0.02}{#1}}
\newcommand{\VariableTok}[1]{\textcolor[rgb]{0.00,0.00,0.00}{#1}}
\newcommand{\VerbatimStringTok}[1]{\textcolor[rgb]{0.31,0.60,0.02}{#1}}
\newcommand{\WarningTok}[1]{\textcolor[rgb]{0.56,0.35,0.01}{\textbf{\textit{#1}}}}
\usepackage{graphicx}
\makeatletter
\def\maxwidth{\ifdim\Gin@nat@width>\linewidth\linewidth\else\Gin@nat@width\fi}
\def\maxheight{\ifdim\Gin@nat@height>\textheight\textheight\else\Gin@nat@height\fi}
\makeatother
% Scale images if necessary, so that they will not overflow the page
% margins by default, and it is still possible to overwrite the defaults
% using explicit options in \includegraphics[width, height, ...]{}
\setkeys{Gin}{width=\maxwidth,height=\maxheight,keepaspectratio}
% Set default figure placement to htbp
\makeatletter
\def\fps@figure{htbp}
\makeatother
\setlength{\emergencystretch}{3em} % prevent overfull lines
\providecommand{\tightlist}{%
  \setlength{\itemsep}{0pt}\setlength{\parskip}{0pt}}
\setcounter{secnumdepth}{-\maxdimen} % remove section numbering
\ifLuaTeX
  \usepackage{selnolig}  % disable illegal ligatures
\fi
\IfFileExists{bookmark.sty}{\usepackage{bookmark}}{\usepackage{hyperref}}
\IfFileExists{xurl.sty}{\usepackage{xurl}}{} % add URL line breaks if available
\urlstyle{same} % disable monospaced font for URLs
\hypersetup{
  pdftitle={EXERCICIOS REVISÃO ATÉ 07/05},
  pdfauthor={WILLIAM IRINEU},
  hidelinks,
  pdfcreator={LaTeX via pandoc}}

\title{EXERCICIOS REVISÃO ATÉ 07/05}
\author{WILLIAM IRINEU}
\date{2023-05-07}

\begin{document}
\maketitle

\hypertarget{aula-4-aceitacao-e-rejeicao}{%
\subsection{AULA 4 ACEITACAO E
REJEICAO}\label{aula-4-aceitacao-e-rejeicao}}

\begin{Shaded}
\begin{Highlighting}[]
\CommentTok{\#exercicio 1 aula 2}

\CommentTok{\#plotando a função beta(3,3)}
\NormalTok{x}\OtherTok{=}\ControlFlowTok{function}\NormalTok{(x)\{}\DecValTok{30}\SpecialCharTok{*}\NormalTok{x}\SpecialCharTok{\^{}}\DecValTok{2}\SpecialCharTok{*}\NormalTok{(}\DecValTok{1}\SpecialCharTok{{-}}\NormalTok{x)}\SpecialCharTok{\^{}}\DecValTok{2}\NormalTok{\}}

\CommentTok{\#funcao}
\FunctionTok{plot}\NormalTok{(x,}\AttributeTok{ylim=}\FunctionTok{c}\NormalTok{(}\DecValTok{0}\NormalTok{,}\DecValTok{3}\NormalTok{),}\AttributeTok{xlim =} \FunctionTok{c}\NormalTok{(}\SpecialCharTok{{-}}\DecValTok{1}\NormalTok{,}\DecValTok{2}\NormalTok{))}
\end{Highlighting}
\end{Shaded}

\includegraphics{exercicio-aula-4_files/figure-latex/setup-1.pdf}

\begin{Shaded}
\begin{Highlighting}[]
\CommentTok{\#funcao entre 0 e 1}
\FunctionTok{plot}\NormalTok{(x)}
\CommentTok{\#plotando linha no x=1/2 (meio) ponto de maximo entre 0 e 1}

\FunctionTok{abline}\NormalTok{(}\AttributeTok{v =} \FloatTok{0.5}\NormalTok{, }\AttributeTok{col =} \StringTok{"red"}\NormalTok{, }\AttributeTok{lwd =} \DecValTok{2}\NormalTok{)}
\end{Highlighting}
\end{Shaded}

\includegraphics{exercicio-aula-4_files/figure-latex/setup-2.pdf}

\begin{Shaded}
\begin{Highlighting}[]
\CommentTok{\#declarando a funcao beta(3,3) dividido pela funcao g, multiplicado pela constante c}

\NormalTok{c}\OtherTok{=}\DecValTok{1}\SpecialCharTok{/}\DecValTok{2} \CommentTok{\#declarando constante}
\NormalTok{fgc}\OtherTok{=}\ControlFlowTok{function}\NormalTok{(x)}
\NormalTok{    \{res}\OtherTok{=}\DecValTok{30}\SpecialCharTok{*}\NormalTok{(x}\SpecialCharTok{\^{}}\DecValTok{2}\NormalTok{)}\SpecialCharTok{*}\NormalTok{(}\DecValTok{1}\SpecialCharTok{{-}}\NormalTok{x)}\SpecialCharTok{\^{}}\DecValTok{2}\SpecialCharTok{*}\NormalTok{c\}}

\CommentTok{\# Algoritmo  }

\FunctionTok{set.seed}\NormalTok{(}\DecValTok{2023}\NormalTok{)}
\NormalTok{n }\OtherTok{\textless{}{-}} \DecValTok{1000}
\NormalTok{x }\OtherTok{\textless{}{-}} \FunctionTok{numeric}\NormalTok{(n) }\CommentTok{\# amostra requerida }
\NormalTok{cont }\OtherTok{\textless{}{-}} \DecValTok{0}  \CommentTok{\# vai contar até atingir o tamanho amostral}
\NormalTok{j }\OtherTok{\textless{}{-}} \DecValTok{0} \CommentTok{\# vai contar as iterações }

\ControlFlowTok{while}\NormalTok{(cont}\SpecialCharTok{\textless{}}\NormalTok{n)}
\NormalTok{\{}
  
\NormalTok{  u }\OtherTok{\textless{}{-}} \FunctionTok{runif}\NormalTok{(}\DecValTok{1}\NormalTok{)}
\NormalTok{  j }\OtherTok{\textless{}{-}}\NormalTok{ j}\SpecialCharTok{+}\DecValTok{1}
\NormalTok{  y }\OtherTok{\textless{}{-}} \FunctionTok{runif}\NormalTok{(}\DecValTok{1}\NormalTok{) }\CommentTok{\# gerando valores da densidade g}
  \ControlFlowTok{if}\NormalTok{(}\FunctionTok{fgc}\NormalTok{(y)}\SpecialCharTok{\textgreater{}}\NormalTok{u)\{}
    
\NormalTok{    cont }\OtherTok{\textless{}{-}}\NormalTok{ cont}\SpecialCharTok{+}\DecValTok{1} 
\NormalTok{    x[cont] }\OtherTok{\textless{}{-}}\NormalTok{ y}
\NormalTok{                \}}
\NormalTok{\}}

\FunctionTok{hist}\NormalTok{(x,}\AttributeTok{prob=}\NormalTok{T, }\AttributeTok{main=}\FunctionTok{expression}\NormalTok{(}\StringTok{"B eta"}\NormalTok{(alpha}\SpecialCharTok{==}\DecValTok{3}\NormalTok{, beta}\SpecialCharTok{==}\DecValTok{3}\NormalTok{)), }\AttributeTok{ylab=}\StringTok{"Densidade"}\NormalTok{,}\AttributeTok{ylim=}\FunctionTok{c}\NormalTok{(}\DecValTok{0}\NormalTok{,}\DecValTok{2}\NormalTok{))   }
\NormalTok{aux }\OtherTok{\textless{}{-}} \FunctionTok{seq}\NormalTok{(}\DecValTok{0}\NormalTok{,}\DecValTok{1}\NormalTok{,}\FloatTok{0.01}\NormalTok{)}
\CommentTok{\# curva da densidade beta(2,2)}
\FunctionTok{lines}\NormalTok{(aux,}\FunctionTok{dbeta}\NormalTok{(aux,}\DecValTok{3}\NormalTok{,}\DecValTok{3}\NormalTok{), }\AttributeTok{col=}\StringTok{"red"}\NormalTok{) }
\end{Highlighting}
\end{Shaded}

\includegraphics{exercicio-aula-4_files/figure-latex/setup-3.pdf}

\begin{Shaded}
\begin{Highlighting}[]
\FunctionTok{cat}\NormalTok{(}\StringTok{"Quant. de iteracoes="}\NormalTok{,j, }\StringTok{"}\SpecialCharTok{\textbackslash{}n}\StringTok{"}\NormalTok{)}
\end{Highlighting}
\end{Shaded}

\begin{verbatim}
## Quant. de iteracoes= 1941
\end{verbatim}

\begin{Shaded}
\begin{Highlighting}[]
\CommentTok{\# Para comparar a amostra obtida por meio dos decis {-} percentis}

\NormalTok{p }\OtherTok{\textless{}{-}} \FunctionTok{seq}\NormalTok{(}\FloatTok{0.1}\NormalTok{,}\FloatTok{0.9}\NormalTok{,}\FloatTok{0.1}\NormalTok{)}
\NormalTok{Dhat }\OtherTok{\textless{}{-}} \FunctionTok{quantile}\NormalTok{(x,p)}
\NormalTok{D }\OtherTok{\textless{}{-}} \FunctionTok{qbeta}\NormalTok{(p,}\DecValTok{3}\NormalTok{,}\DecValTok{2}\NormalTok{)}

\FunctionTok{round}\NormalTok{(}\FunctionTok{rbind}\NormalTok{(Dhat, D),}\DecValTok{3}\NormalTok{)}
\end{Highlighting}
\end{Shaded}

\begin{verbatim}
##        10%   20%   30%   40%   50%   60%   70%   80%   90%
## Dhat 0.239 0.318 0.387 0.439 0.488 0.540 0.605 0.664 0.749
## D    0.320 0.418 0.492 0.555 0.614 0.671 0.728 0.788 0.857
\end{verbatim}

\end{document}
